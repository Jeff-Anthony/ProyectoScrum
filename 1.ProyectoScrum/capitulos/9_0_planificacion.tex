\chapter{Planificación}

\begin{table}[htb]
\centering
\begin{tabular}{|l|p{1.5cm}|p{10cm}|}
	\hline
	\multicolumn{3}{|c|}{Tareas} \\ \hline
	Id & Código & Tarea \\
	\hline \hline
	1 & H01  & Registro:\begin{itemize}
		\item Creación de la Base de datos.
			\begin{itemize}
			\item Creación de tabla:
			 		\begin{itemize}
			 		\item Nombres*
			 		\item Apellidos*
			 		\item DNI
			 		\item Correo*
			 		\item Edad*
			 		\item Alergia		 	 
			 	\end{itemize}
			\end{itemize}
		\item Creación de la Interfaz
		
		
	\end{itemize}\\ \hline
	
	2 & H02  & Logeo: \begin{itemize}
		\item Conexión a la Base de datos con los atributos:							
			\begin{itemize}						
				\item Correo
				\item Contraseña	 	 				
			\end{itemize}
		
		\item Creación de la Interfaz
		\end{itemize}\\ \hline
	
	3 & H03 & Interfaz de enfermedades y medicamentos: 
	Como usuario, debería poder tener acceso a la información de medicamentos según la enfermedad que he consultado.   \\ \hline
	
	4 & H04  & Historial:Como usuario, debería poder llevar un seguimiento de las enfermedades y medicamentos consultadas \\ \hline
	
	5 & H05  & Como desarrollador frontend yo debería usar los 
	principios de UX para tener una interfaz más amigable 
	y eficiente para el usuario. 
	\\ \hline
	
	
\end{tabular}
\caption{Tabla Historias de usuario.}
\label{tabla:Historial}
\end{table}