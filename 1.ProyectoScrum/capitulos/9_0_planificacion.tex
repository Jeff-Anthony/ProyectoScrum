\chapter{Planificación}

\begin{table}[htb]
\centering
\begin{tabular}{|l|p{1.5cm}|p{2.5cm}|p{10cm}|}
	\hline
	\multicolumn{4}{|c|}{Tareas} \\ \hline
	Id & Código & Responsables & Tarea \\
	\hline \hline
	1 & H01 & Vilchez & Registro:\begin{itemize}
		\item Creación de la Base de datos.
			\begin{itemize}
			\item Creación de tabla:
			 		\begin{itemize}
			 		\item Nombres*
			 		\item Apellidos*
			 		\item DNI
			 		\item Correo*
			 		\item Edad*
			 		\item Alergia		 	 
			 	\end{itemize}
			\end{itemize}
		\item Creación de la Interfaz
		
		
	\end{itemize}\\ \hline
	
	2 & H02 & Vilchez & Logeo: \begin{itemize}
		\item Conexión a la Base de datos con los atributos:							
			\begin{itemize}						
				\item Correo
				\item Contraseña	 	 				
			\end{itemize}
		
		\item Creación de la Interfaz
		\end{itemize}\\ \hline
	
	3 & H03 & Llanos y Antonio & Interfaz de enfermedades y medicamentos: 

	 \begin{itemize}
		\item Creación de la base de datos para los medicamentos, enfermedades y su información.
		\item Creación del sistema de búsqueda por enfermedades.
		\item Creación de una lista de medicamentos según la enfermedad escogida.
		\item Creación de las 3 secciones de medicamentos (Generales, Laboratorio y Naturales).
		\item Creación de indices para búsquedas más eficientes y reducir la sobre carga de información.
		\item Creación de la Interfaz
	\end{itemize}\\ \hline

	
	4 & H04 & Antonio & Historial:
	\begin{itemize}
		\item Registro de búsquedas de enfermedades y medicamentos seleccionados.
		\item Creación de la interfaz
	\end{itemize}\\ \hline
	
	
\end{tabular}
\caption{Tabla Tareas.}
\label{tabla:Tareas}
\end{table}