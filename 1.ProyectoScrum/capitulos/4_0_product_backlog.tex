\chapter{Product Backlog}


\begin{table}[htb]
	\centering
	\begin{tabular}{|l|p{2cm}|l|p{9cm}|l|}
		\hline
		\multicolumn{5}{|c|}{Backlog} \\ \hline
		Id & Código & Prioridad & Descripción & Estado \\
		\hline \hline
	
		1 & PB01 & 20 & Registro: Se muestra un formulario para que el usuario pueda rellenar la información pedida y completar el registro de su cuenta. & Hecho \\ \hline
		2 & PB02 & 30 & Inicio de sesión: Se podrá proceder a iniciar sesión una vez ya se haya completado el registro, pudiendo así ingresar a la plataforma de búsqueda de medicamentos por enfermedades. & Hecho \\ \hline
		
		3 & PB03 & 40 & Interfaz de enfermedades y medicamentos: Habrá un buscador que tendrá diversos tipos de enfermedades según el usuario desee saber. También 3 botones tipo carpetas que mostrarán las medicinas generales ,de laboratorio y naturales según la enfermedad escogida. & Sin hacer \\ \hline
		4 & PB04 & 10 & Historial: Registro de enfermedades y medicamentos que se han consultado al consultar en el buscador. & Sin hacer \\ \hline
	\end{tabular}
	\caption{Tabla Backlog.}
	\label{tabla:Backlog}
\end{table}
