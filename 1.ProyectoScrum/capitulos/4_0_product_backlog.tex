\chapter{Product Backlog}

\section{Primera}

\begin{table}[htb]
	\centering
	\begin{tabular}{|l|p{2cm}|l|p{9cm}|l|}
		\hline
		\multicolumn{5}{|c|}{Backlog} \\ \hline
		Id & Código & Prioridad & Descripción & Estado \\
		\hline \hline
		1 & PB01 & 20 &  Conexión a la base de datos :Se crearán  las tablas para el registro de usuarios,de medicamentos con sus imágenes y enfermedades específicas. & Sin hacer \\ \hline
		2 & PB02 & 30 & Registro: Habrá una vista móvil con los campos a rellenar por el usuario, los cuales son: Nombres ,apellidos,DNI y edad. & Sin hacer \\ \hline
		3 & PB03 & 30 & Logeo: El usuario podrá entrar a la plataforma, a fin de poder llevar un tratamiento con los medicamentos apropiados. & Sin hacer \\ \hline
		
		4 & PB04 & 30 & Interfaz de enfermedades y medicamentos: Habrá un buscador que tendrá diversos tipos de enfermedades según el usuario desee saber. También 3 botones tipo carpetas que mostrarán las medicinas generales ,de laboratorio y naturales según la enfermedad escogida. & Sin hacer \\ \hline
		5 & PB05 & 20 & Historial: Registro de enfermedades y medicamentos que se han consultado. & Sin hacer \\ \hline
	\end{tabular}
	\caption{Tabla muy sencilla.}
	\label{tabla:Backlog}
\end{table}
